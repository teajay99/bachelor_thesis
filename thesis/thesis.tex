%!TeX spellcheck = en-DE

\documentclass[a4paper,twoside]{scrartcl}
\usepackage{packages}

\newcommand{\SUTwo}{$\mathrm{SU}(2)$\xspace}

\tabulinesep=1.2mm

\begin{document}

\input{titlepage}

\begin{abstract}
 In this thesis we will take a look at different approaches to approximating the group \SUTwo by a finite subset in its function as the gauge group of a quantum field theory. This includes finite subgroups of \SUTwo like the popular icosahedral approximation, four-dimensional versions of the Platonic solids and geodesic polytopes, as well as a generalized Fibonacci lattice. In order to test these approximations, Metropolis Monte Carlo simulations were implemented and run.
\end{abstract}

\tableofcontents
\newpage

\section{Introduction}

\import{chap_introduction/}{chapter.tex}

\section{The Lattice Action}
\import{chap_lattice_gauge/}{chapter.tex}

\section{Discretization of \SUTwo}
\label{sec:discSu2}
\import{chap_discrete_su_2/}{chapter.tex}

\section{Metropolis Monte Carlo Method}
\import{chap_monte_carlo/}{chapter.tex}

\section{Results}
\label{sec:results}
\import{chap_results/}{chapter.tex}

\section{Conclusion}
\label{sec:conclusion}
\import{chap_conclusion/}{chapter.tex}

\printbibliography


\end{document}
