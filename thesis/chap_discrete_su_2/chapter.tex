A general Element $U \in \mathrm{SU}(2)$ in the fundamental representation can be written as
\begin{align*}
 U & =
 \begin{pmatrix}
  u             & w            \\
  -\overline{w} & \overline{u}
 \end{pmatrix} \qquad \textrm{mit} \quad u,w \in \mathbb{C} \quad \textrm{und} \quad \abs{u} + \abs{w} = 1
\end{align*}
\cite{findsource}. Somit ist $\mathrm{SU}(2)$ diffeomorph zu $S^3$, der Einheitssphäre im $\mathbb{R}^4$, und equivalent zur Gruppe der Einheitsquaternionen
\begin{align*}
 \mathbb{H}_1 = \left\{ x \in \mathbb{H} : \norm{x} = 1\right\} \textrm{.}
\end{align*}
\cite{findsource}. Zudem ist $\mathrm{SU}(2)$ die zweifache Überlagerung der Drehgruppe $\mathrm{SO}(3)$ in $\mathbb{R}^3$ \cite{findsource}. Im folgenden sollen diese Eigenschaften genutzt werden, um representative endliche Untermengen von $\mathrm{SU}(2)$ zu finden.

\subsection{Finite Subgroups of \SUTwo}

The first thing to take a look at, in terms of representative finite subsets of \SUTwo are its finite subgroups. As first proven in \cite{findsource} these can be constructed by taking the cartesian product of $\mathbb{Z}_2$ with the subgroups of $\mathrm{SO}(3)$. The finite subgroups of $\mathrm{SO}(3)$ can be constructed by all the symmetry transformations of regular polygons as well as rotations of the platonic solids \cite{findsource}. The groups generated by the regular polygons where however not considered in this work, as they are restricted to a plane. Therefore it is unlikely that these will give a working \emph{representation} of \SUTwo.\\

The five platonic solids give rise to three distinct subgroups. The first one would be the tetrahedral group $T$ with $12$ elements, which as the name suggest is generated by the orientation preserving rotations of a Tetrahedron. Next up is the octahedral group $O$ with $24$ elements, generated by the rotational symmetries of the Octahedron ( or the Cube respectively as it is the geometric dual of the Octahedron). Finally there is the icoshedral group $I$ with $60$ elements generated by the Icosahedron/Dodecahedron.

Expanding these groups to \SUTwo doubles the element counts and therefore gives rise to the so called binary tetrahedral group $\overline{T}$, the binary octahedral group $\overline{O}$ and the binary icoshedral group $\overline{I}$, with $24$, $48$ and $120$ elements respectively. One possible representation can be found in table \ref{tab:subgroups}.

\begin{table}
 \centering
 \begin{tabular}{c|c|c}
  group          & order & elements                                                                                                                                        \\
  \hline
  $\overline{T}$ & 24          & all sign combinations of $\left\{\pm 1, \pm i, \pm j, \pm k, \frac{1}{2}\left( \pm 1 \pm i \pm j \pm k \right) \right\} $                       \\
  \hline
  $\overline{O}$ & 48          & all sign combinations and permutations of                                                                                                       \\
                 &             & $\left\{\pm 1, \pm i, \pm j, \pm k, \frac{1}{2}\left( \pm 1 \pm i \pm j \pm k \right) , \frac{1}{\sqrt{2}}\left( \pm 1 \pm i \right) \right\} $ \\
  \hline
  $\overline{I}$ & 120         & all sign combinations and even permutations of of                                                                                                  \\
                 &             & $\left\{\pm 1, \pm i, \pm j, \pm k, \frac{1}{2}\left( \pm 1 \pm i \pm j \pm k \right) , \frac{1}{2}\left(1+\tau i + \frac{j}{\tau} \right) \right\} $                   \\
 \end{tabular}
 \caption{Quaternionic Representation of $\overline{T}$, $\overline{O}$ and $\overline{I}$ as found in \cite{duval:1964}, where $\tau= \frac{1+\sqrt{5}}{\sqrt{2}}$ denotes the golden ratio}
 \label{tab:subgroups}
\end{table}

\subsection{Regular Polytypes in four dimensions}

The other nice thing about the platonic solids, is that their vertices are a well distributed set of points on $S_2$. As \SUTwo is diffeomorphic to $S_3$ a natural next step would therefore be to take a look at the regular polytopes in four dimensions. In total there are six of those, with vertex counts ranging from the 5 vertices of the 5-Cell to the 600 vertices of the 120-cell.

\begin{table}
  \centering
  \begin{tabular}{c|c|c}

  \end{tabular}
  \caption{title}
\end{table}

\subsection{Geodesic Polytypes}

\subsection{Fibonacci Gitter auf $\mathrm{SU}(2)$}

\subsection{Quellen}

https://journals.aps.org/prd/pdf/10.1103/PhysRevD.27.412

https://journals.aps.org/prd/pdf/10.1103/PhysRevD.22.2465


\begin{itemize}
 \item Naive Partitionierung von $S_2$ und $S_3$
 \item Ikosaeder in 4 Dimensionen und abgefahrenes Zeug was man damit anstellen kann $H_4$ (\url{https://arxiv.org/pdf/1705.04910.pdf})
 \item Fibonacci Latice (\url{https://stackoverflow.com/questions/57123194/how-to-distribute-points-evenly-on-the-surface-of-hyperspheres-in-higher-dimensi} and \url{https://math.stackexchange.com/questions/3291489/can-the-fibonacci-lattice-be-extended-to-dimensions-higher-than-3/3297830#3297830})
\end{itemize}
SO(3) Untergruppen: http://math.uchicago.edu/~may/REU2020/REUPapers/Bui,An.pdf
