A general Element $U \in \mathrm{SU}(2)$ in the fundamental representation can be written as
\begin{align}
 U & =
 \begin{pmatrix}
  u             & w            \\
  -\overline{w} & \overline{u}
 \end{pmatrix} \qquad \textrm{with} \quad u,w \in \mathbb{C} \quad \textrm{and} \quad \mathrm{det} \, U = \abs{u} + \abs{w} = 1 \textrm{.}
 \label{eq:generalSu2Element}
\end{align}
The group composition in this representation is simply implemented by matrix multiplication. If one sets $u = a + ib$ and $w = c + id$ the determinant constraint implies that the point $(a,b,c,d)^T$ lies on the unit sphere $S_3$ embedded in $\mathbb{R}^4$. In order to obtain the group composition in terms of the real parameters $(a,b,c,d)$ one simply carries out the matrix multiplication $U'' = U U'$ for two points $U \equiv (a,b,c,d)$ and $U' \equiv (a',b',c',d')$:
\begin{align*}
 a'' & = a a' - b b' - c c' - d d'            \\
 b'' & = a b' + b a' + c d' - d c'            \\
 c'' & = a c' - b d' + c a' + d b'            \\
 d'' & = a d' + b c' - c b' + d a' \textrm{.}
\end{align*}
In order to represent this group composition more conveniently one can introduce the Quaternions $\mathbb{H}$, which will be the notation used in the following chapter. They form an extension of the complex numbers to four dimensions. The Basis is usually notated as $(1,i,j,k)$ with the non commutative multiplication defined by
\begin{align*}
 i^2 = j^2 = k^2 = ijk = -1 \textrm{.}
\end{align*}
Therefore the group composition can be implemented by the Quaternion product $q'' = q q'$ for two Quaternions $q = a + bi + cj + dk$ and $q' = a' + b'i + c'j + d'k$. One can furthermore introduce the norm $\norm{q} = \sqrt{a^2 + b^2 + c^2 +d^2}$ similar to the euclidean norm of $\mathbb{R}^4$ constraining elements of \SUTwo to the unit quaternions $\mathbb{H}_1$:
\begin{align*}
 \mathbb{H}_1 = \left\{ x \in \mathbb{H} : \norm{x} = 1\right\} \textrm{.}
\end{align*}
To implement the previously discussed group action in terms of unit quaternions we also need to be able to calculate the traces of group elements. This can be achieved by
\begin{align*}
 \mathrm{Tr}\left[ U \right] & = \mathrm{Tr}\left[ \begin{pmatrix}
   a + ib  & c + id \\
   -c + id & a - ib \\
  \end{pmatrix} \right] = 2 a = : 2 \mathrm{Re} (q) \textrm{.}
\end{align*}
In the following different approach es to finding well distributed finite subsets of $\mathbb{H}_1$ will be discussed. The goal here is to approximate \SUTwo in its function as a gauge group, which is why theses subsets will be referred to as gauge sets. Explicitly this means the Link Variables $U_\mu$ will be restricted to members of these subsets. The main metric for success is then given by the results obtained in chapter \ref{sec:results}, where numerical simulations of the gauge theory will be conducted.

\subsection{Finite Subgroups of \SUTwo}

The first thing to take a look at are the finite subgroups of \SUTwo. As first proven in \cite{findsource} these can be constructed by taking the cartesian product of the cyclic group of order 2 with the subgroups of $\mathrm{SO}(3)$. The finite subgroups of $\mathrm{SO}(3)$ can be constructed by all the symmetry transformations of regular polygons as well as rotations of the platonic solids \cite{findsource}. The groups generated by the regular polygons where however not considered in this work, as they are restricted to planes within $\mathbb{R}^4$. Therefore they will most likely not be a good approximation of \SUTwo.\\

The five platonic solids give rise to three distinct subgroups. The first one would be the tetrahedral group $T$ with $12$ elements, which as the name suggest is generated by the orientation preserving rotations of a Tetrahedron. Next up is the octahedral group $O$ with $24$ elements, generated by the rotational symmetries of the Octahedron (or the Cube respectively as it is the geometric dual of the Octahedron). Finally there is the icoshedral group $I$ with $60$ elements generated by the rotational symmetries the Icosahedron and Dodecahedron.

Expanding these groups to \SUTwo doubles the element counts and therefore gives rise to the so called binary tetrahedral group $\overline{T}$, the binary octahedral group $\overline{O}$ and the binary icoshedral group $\overline{I}$, with $24$, $48$ and $120$ elements respectively. The quaternionic representation used in the following can be found in table \ref{tab:subgroups}.

\begin{table}
 \centering
 \begin{tabu}{c|c|l}
  group          & order & elements                                                                                                                                                   \\
  \hline
  $\overline{T}$ & 24    & all sign combinations of $\left\{\pm 1, \pm i, \pm j, \pm k, \frac{1}{2}\left( \pm 1 \pm i \pm j \pm k \right) \right\} $                                  \\
  \hline
  $\overline{O}$ & 48    & all sign combinations and permutations of                                                                                                                  \\
                 &       & $\left\{\pm 1, \pm i, \pm j, \pm k, \frac{1}{2}\left( \pm 1 \pm i \pm j \pm k \right) , \frac{1}{\sqrt{2}}\left( \pm 1 \pm i \right) \right\} $            \\
  \hline
  $\overline{I}$ & 120   & all sign combinations and even permutations of of                                                                                                          \\
                 &       & $\left\{\pm 1, \pm i, \pm j, \pm k, \frac{1}{2}\left( \pm 1 \pm i \pm j \pm k \right) , \frac{1}{2}\left(1+\tau i + \frac{j}{\tau} + 0k \right) \right\} $ \\
 \end{tabu}
 \caption{Quaternionic Representation of $\overline{T}$, $\overline{O}$ and $\overline{I}$ as found in \cite{duval:1964}, where $\tau= \frac{1+\sqrt{5}}{\sqrt{2}}$ denotes the golden ratio}
 \label{tab:subgroups}
\end{table}

\subsection{Regular Polytopes in four dimensions}

Another useful property of platonic solids, is that their vertices are, if scaled accordingly, a well distributed set of points on $S_2$. As we are trying to distribute points on $S_3$, there is a good chance the vertices of something like a four dimensional platonic solid would make for a useful gauge set. For higher dimensions than three one refers to such contraptions as a regular polytope, and there are six of these in four dimension. They are denoted by the number of cells (i.e. three dimensional faces) they have. The vertex count ranges from the 5 vertices of the 5-cell $C_5$ to the 600 vertices of the 120-cell $C_{120}$. The vertices of all six can be found in table \ref{tab:polytopes}.

It is noteworthy that the vertices of $C_{24}$ and  $C_{600}$ coincide with $\overline{T}$ and $\overline{I}$ respectively. The vertices of $C_{16}$ also form a group called the binary dihedral group $\overline{D}_4$ \cite{findsource}.

\begin{table}
 \centering
 \begin{tabu}{c|c|l}
  polytope  & \# vertices & vertices                                                                                                                                    \\
  \hline
  $C_5$     & 5           & $\begin{array}{l}
    \left\{ 1,  -\frac{1}{4} + \eta i + \eta j + \eta k,  -\frac{1}{4} + \eta i + \eta j  + \eta k, \right. \\
    \left.  -\frac{1}{4} + \eta i + \eta j + \eta k, -\frac{1}{4} + \eta i + \eta j + \eta k \right\}
   \end{array}$                                                                                                                 \\
  \hline
  $C_{16}$  & 8           & $\left\{ \pm 1, \pm i, \pm j, \pm k \right\}$                                                                                               \\
  \hline
  $C_8$     & 16          & all sign combinations of $ \left\{  \frac{1}{2} (\pm 1 \pm i \pm j \pm k)\right\}$                                                          \\
  \hline
  $C_{24}$  & 24          & $C_8 \cup C_{16}$ (same as $\overline{T}$)                                                                                                  \\
  \hline
  $C_{600}$ & 120         & \makecell[l]{ $ C_{24} \cup \left\{  \textrm{sign comb. and even perm. of } \frac{1}{2} \left(1+\tau i + \frac{j}{\tau} + 0k \right)  \right\} $ \\ (same as $\overline{I}$)} \\
  \hline
  $C_{120}$ & 600         & $\{a b \, | \, a \in C_5, \, b \in C_{600} \}$                                                                                              \\
 \end{tabu}
 \caption{title}
 \label{tab:polytopes}
\end{table}

\subsection{Geodesic Polytopes in Four Dimensions}

Furthermore we can make use of the fact, that the faces of the platonic solids are regular polygons. This allows for the construction of so called geodesic polyhedra \ref{findsource}. As seen in figure \ref{fig:icospherePic} the basic idea here is to subdivide the faces of an platonic solid into triangles and then project the obtained vertices onto the sphere. Although the cells of the regular polytopes in four dimensions are the platonic solids, such a procedure is complicated by the fact that most of the platonic solids do not tile space on their own. Therefore finding appropriate subdivision in a scaleable way is not a trivial task.\\

\begin{figure}
 \centering
 \begin{subfigure}{0.9\textwidth}
  \includegraphics[width=\textwidth]{icosphere.png}
  \caption{}
  \label{fig:icospherePic}
 \end{subfigure}
 \begin{subfigure}{0.9\textwidth}
  \includegraphics[width=\textwidth]{volleyball.png}
  \caption{}
  \label{fig:volleyPic}
 \end{subfigure}
 \caption{title}
\end{figure}

The only exception to this would be $C_8$, as its eight cells are cubes. In this case any of the eight \emph{surface cubes} can be trivially filled with $m^3$ cubes of sidelength $\frac{1}{m}$ for $m \in \mathbb{N}/\{0\}$. As seen in figure \ref{fig:volleyPic} the grid obtained by this procedure in three dimensions is somewhat reminiscent of a volleyball. So in the following we will denote such a \emph{volleyball lattice} with n subdivision by $V_n$. \\

As any of the eight \emph{surface cubes} is fixed by holding one of the coordinates at $\pm 1$ the set of coordinates before projection onto the sphere $\tilde{V}_n$ is obtained by
\begin{align*}
 \tilde{V}_n & = \left\{ \textrm{all permutations of }  \, \frac{1}{2} \left( \pm 1 + v_n(a) i + v_n(b) j + v_n(c) k \right)  \, \middle| \, a,b,c \in \mathbb{N}: 0 \le a,b,c \le n+1 \right\}
\end{align*}
with $v_n$ beeing defined as
\begin{align*}
 \quad v_n(m)  =  1-\frac{2m}{n+1} \, \textrm{.}
\end{align*}
$V_n$ is then simply obtained by dividing out the norm for every point in $\tilde{V}_n$:
\begin{align*}
 V_n = \left\{\frac{q}{\norm{q}} \, \middle| \,  q \in \tilde{V}_n \right\} \, \textrm{.}
\end{align*}
The number of distinct elements of $V_n$ can be calculated to be
\begin{align*}
 \abs{V_n} & = \underbrace{16}_{\substack{ \textrm{vertices of } C_8
 }} + \underbrace{8 n^3}_{\substack{n^3 \, \textrm{vertices added in} \\ \textrm{every one of the 8 cells}}} + \underbrace{24 n^2}_{\substack{n^2 \, \textrm{vertices added on} \\ \textrm{every one of the 24 faces}}} + \underbrace{32 n}_{\substack{n \, \textrm{vertices added on} \\ \textrm{every one of the 32 edges}}} \, \textrm{.}
\end{align*}
The main advantage of the \emph{volleyball} lattice is that one can vary the number of subdivisions, and therefore allows some control over the granularity of the gauge set. Due to the method of construction the density of vertices will however increase towards the vertices of $C_8$. This will probably lead to some systematic deviations when using $V_n$ as an approximation of \SUTwo.

\subsection{Fibonacci lattice on \SUTwo}

The final discretization of \SUTwo considered in this work is a higher dimensional version of the so called Fibonacci lattice. Fibonacci Lattices offer an elegant and deterministic solution to to the problem of distributing a given amount of points on a two dimensional surface. They are used in numerous fields of research such as numerical analysis or computer graphics, mostly to approximate spheres (as e.g. shown in figure \ref{fig:fibonacciPic}). Mainly inspired by \ref{stack:fibonacci} we will now attempt to construct a similar lattice for $S_3$.\\
\begin{figure}
 \centering
 \includegraphics[width=0.8\textwidth]{fibonacci3d.png}
 \caption{title}
 \label{fig:fibonacciPic}
\end{figure}

\noindent The two dimensional fibonacci lattice is usually constructed within a unit square $[0,1)^2$ as
\begin{align*}
 \tilde{L}_n                      & = \left\{ \tilde{t}_m \middle| 0 \le m < n, \, \, m \in \mathbb{N} \right\}                                    \\
 \textrm{with} \qquad \tilde{t}_m & = \begin{pmatrix}x_m\\y_m\end{pmatrix} = \left(\frac{1}{\tau} \quad \mathrm{mod} \quad 1, \frac{m}{n} \right) \textrm{.}
\end{align*}
\noindent This can be generalized to the hypercube $[0,1)^N$ embedded in $\mathbb{R}^N$.
\begin{align*}
 L_n & = \left\{ t_m \middle| 0 \le m < n, \, \, m \in \mathbb{N} \right\}                                                                                       \\
 t_m & = \begin{pmatrix} t_m^1 \\ t_m^2 \\ \vdots \\ t_m^N \end{pmatrix} = \begin{pmatrix}
  \frac{m}{n}        &                      \\
  a_1 \, m \quad     & \mathrm{mod} \quad 1 \\
  \vdots             &                      \\
  a_{N-1} \, m \quad & \mathrm{mod} \quad 1 \\
 \end{pmatrix} \qquad \textrm{with} \qquad \frac{a_i}{a_j} \notin \mathbb{Q} \quad \textrm{for} \quad i \neq j
\end{align*}
A simple choice for the constants $a_i$ would be the square roots of the prime numbers:
\begin{align*}
 (a_1, a_2 ,a_3, \dots) = (\sqrt{2}, \sqrt{3}, \sqrt{5}, \dots)
\end{align*}
The points in $L_n$ are then evenly distributed within the given Volume. All that is left to do is to map these points onto a given compact manifold $M$, in our case \SUTwo. In order to maintain the even distribution of the points such a map $\Phi$ needs to be volume preserving in the sense that
\begin{align}
 \int_{\Omega \subseteq [0,1)^N} \mathrm{d}^N x = \frac{1}{\mathrm{Vol}(M)} \int_{\Phi(\Omega) \subseteq M} \mathrm{d}V_M
 \label{eq:fibAreaPres}
\end{align}
holds for all regions $\Omega$.

In order to find such a map for $S_3$ (and therefore \SUTwo) we start by embedding $S_3$ into $\mathbb{H}$ by introducing spherical coordinates
\begin{align*}
 z (\psi, \theta, \phi) & =
 \cos \psi + \sin \psi \cos \theta \, i + \sin \psi \sin \theta \cos \phi \, j + \sin \psi \sin \theta \sin \phi \, k                                \\
                        & \quad \textrm{with} \quad \psi \in [0,\pi), \quad \theta \in [0,\pi) \quad \textrm{and} \quad \phi \in [0,2\pi) \textrm{.}
\end{align*}
Therefore the metric tensor $g_{ij}$ in terms of the spherical coordinates $(y_1, y_2, y_3) := (\psi, \theta, \phi)$, treating $(1,i,j,k)$ as a euclidean basis, is given by
\begin{align*}
 g_{ij} & = \frac{\partial z^a}{\partial y^i} \frac{\partial z^b}{\partial y^j} \delta_{ab} = \begin{pmatrix}
  1 & 0           & 0                         \\
  0 & \sin^2 \psi & 0                         \\
  0 & 0           & \sin^2 \psi \sin^2 \theta \\
 \end{pmatrix}_{ij} \textrm{.}
\end{align*}
From this one can calculate the jacobian $\sqrt{\abs{g}}$ to be
\begin{align*}
 \sqrt{\abs{g}} & = \sin^2 \psi \sin \theta \textrm{.}
\end{align*}
As $\sqrt{\abs{g}}$ factorizes nicly into functions only dependent on one coordinate, one can construct a bijective map $\Phi^{-1}$ mapping $S_3$ to $[0,1)^3$ given by $\Phi^{-1}(\psi,\theta,\phi) = \left(\Phi^{-1} (\psi), \Phi^{-1}(\theta), \Phi^{-1} (\phi) \right)$ with
\begin{alignat*}{2}
 \Phi^{-1} (\psi)   & = \frac{\int_0^{\psi}\textrm{d}\tilde{\psi} \sin^2 \tilde{\psi}}{\int_0^\pi \textrm{d} \tilde{\psi} \sin^2 \tilde{\psi}} &  & = \frac{1}{\pi}  \left( \psi - \frac{1}{2} \sin( 2 \psi) \right) \\
 \Phi^{-1} (\theta) & = \frac{\int_0^{\theta}\textrm{d}\tilde{\theta} \sin \theta}{\int_0^\pi \textrm{d}\tilde{\theta} \sin \theta}            &  & = \frac{1}{2} \left( 1-\cos(\theta) \right)                      \\
 \Phi^{-1} (\phi)   & = \frac{\int_0^{\phi}\textrm{d}\tilde{\phi} }{\int_0^{2 \pi} \textrm{d} \tilde{\phi}}                                    &  & = \frac{1}{2 \pi} \phi \textrm{.}
\end{alignat*}
Looking at some region $\Omega = \Phi^{-1} (\tilde{\Omega})$ one can see that the inverse map $(\Phi^{-1})^{-1} := \Phi$ trivially fulfills equation \ref{eq:fibAreaPres}. A fibonacci like lattice on $S_3$ would therefore be given by
\begin{align*}
 F_n & = \left\{ z\left(\psi_m(t_m^1), \theta_m(t_m^2), \phi_m(t_m^3)\right)  \middle| \, 0 \le m < n, \, \, m \in \mathbb{N} \right\}
\end{align*}
\begin{alignat*}{2}
 \textrm{with} \quad \psi_m(t_m^1) & =  f^{-1} \left( t_m^1 \right)  &  & = f^{-1} \left( \frac{m\pi}{n+1}\right), \qquad f(x) = x - \frac{1}{2} \sin(2x) ,                                                                     \\
 \theta_m(t_m^2)                   & =  \cos^{-1}\left(t_m^2 \right) &  & =  \cos^{-1}\left(1-2(m\sqrt{2} \quad \mathrm{mod} \quad 1), \right)                                                                                  \\
 \textrm{and} \quad \phi_m (t_m^3) & =  2 \pi t^3_m                  &  & =                        2 \pi (m\sqrt{3} \quad \mathrm{mod} \quad 1)                                                                      \textrm{.} \\
\end{alignat*}
The fibonacci approach offers two important advantages, compared to the previously proposed discretizations. First of all it can generate a representative gauge set for any given number of elements. As the motivation for these discretizations is to be used for simulations on quantum computers, this effectively means that one will always be able to achieve maximum granularity for a given amount of quantum memory \footnote{It should be noted, that the algorithms for such simulations of \SUTwo gauge theorys are still in the works, and therefore way beyond the scope this work. The stated advantage is therefore based on a reasonable assumption and not a know fact.}.

The way the four dimensional version was constructed also suggest, that one can obtain similar procedures for other compact manifolds and therefore other continuous gauge groups. To put this statement into rigorous mathematical terminology or to provide a proof is the scope of this work.

However as long as one manages to introduce coordinates $(y_1, ..., y_n)$ on the manifold, and the metric $g$, obtained by a pullback from $n+1$ dimensional euclidian space, takes a form s.t. $\abs{g} = \prod_{i} \, f_i(y_i)$, one should be able to obtain a amp similar to $\Phi^{-1}$. Therefore there is at least a good chance that such lattices can e.g. be constructed for $\textrm{SU}(3)$ and therefore be used for QCD simulations.

The only disadvantage might be that in spite of the uniform the distribution of the generated points, some residual bias might be left. This could lead to some systematic deviations showing for smaller lattice sizes $n$. In contrast to the volleyball lattice these should however get smaller for bigger $n$.
%Add disadvantages

% \subsection{Quellen}
%
% https://journals.aps.org/prd/pdf/10.1103/PhysRevD.27.412
%
% https://journals.aps.org/prd/pdf/10.1103/PhysRevD.22.2465
%
%
% \begin{itemize}
%  \item Naive Partitionierung von $S_2$ und $S_3$
%  \item Ikosaeder in 4 Dimensionen und abgefahrenes Zeug was man damit anstellen kann $H_4$ (\url{https://arxiv.org/pdf/1705.04910.pdf})
%  \item Fibonacci Latice (\url{https://stackoverflow.com/questions/57123194/how-to-distribute-points-evenly-on-the-surface-of-hyperspheres-in-higher-dimensi} and \url{https://math.stackexchange.com/questions/3291489/can-the-fibonacci-lattice-be-extended-to-dimensions-higher-than-3/3297830#3297830})
% \end{itemize}
% SO(3) Untergruppen: http://math.uchicago.edu/~may/REU2020/REUPapers/Bui,An.pdf
