Ein allgemeines Element $U \in \mathrm{SU}(2)$ in der fundamentalen Representation kann geschrieben werden als
\begin{align*}
 U & =
 \begin{pmatrix}
  u             & w            \\
  -\overline{w} & \overline{u}
 \end{pmatrix} \qquad \textrm{mit} \quad u,w \in \mathbb{C} \quad \textrm{und} \quad \abs{u} + \abs{w} = 1
\end{align*}
\cite{findsource}. Somit ist $\mathrm{SU}(2)$ diffeomorph zu $S^3$, der Einheitssphäre im $\mathbb{R}^4$, und equivalent zur Gruppe der Einheitsquaternionen
\begin{align*}
 \mathbb{H}_1 = \left\{ x \in \mathbb{H} : \norm{x} = 1\right\} \textrm{.}
\end{align*}
\cite{findsource}. Zudem ist $\mathrm{SU}(2)$ die zweifache Überlagerung der Drehgruppe $\mathrm{SO}(3)$in $\mathbb{R}^3$ \cite{findsource}. Im folgenden sollen diese Eigenschaften genutzt werden, um representative endliche Untermengen von $\mathrm{SU}(2)$ zu finden.

\subsection{Endliche Untergruppen der $\mathrm{SU}(2)$}

Als representative Untermenge bieten sich zunächst die endlichen Untergruppen von $\mathrm{SU}(2)$ an. Diese können klassifiziert werden als zyklische Gruppen $\mathrm{Z}$

\subsection{Reguläre konvexe Polyeder in }

\subsection{Fibonacci Gitter auf $\mathrm{SU}(2)$}

\subsection{Quellen}

https://journals.aps.org/prd/pdf/10.1103/PhysRevD.27.412

https://journals.aps.org/prd/pdf/10.1103/PhysRevD.22.2465


\begin{itemize}
 \item Naive Partitionierung von $S_2$ und $S_3$
 \item Ikosaeder in 4 Dimensionen und abgefahrenes Zeug was man damit anstellen kann $H_4$ (\url{https://arxiv.org/pdf/1705.04910.pdf})
 \item Fibonacci Latice (\url{https://stackoverflow.com/questions/57123194/how-to-distribute-points-evenly-on-the-surface-of-hyperspheres-in-higher-dimensi} and \url{https://math.stackexchange.com/questions/3291489/can-the-fibonacci-lattice-be-extended-to-dimensions-higher-than-3/3297830#3297830})
\end{itemize}
SO(3) Untergruppen: http://math.uchicago.edu/~may/REU2020/REUPapers/Bui,An.pdf
