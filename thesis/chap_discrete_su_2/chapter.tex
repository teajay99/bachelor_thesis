
\subsection{Ideen und Notizen (To be removed)}

\subsubsection{Mögliche Quellen}



\subsubsection{Fragen (To be removed)}
\begin{itemize}
  \item Müssen die Partitionierung der Gruppe abgeschlossen unter der Gruppenverknüpfung sein?
\end{itemize}

\subsection{Endliche Untergruppen der $\mathrm{SU}(2)$}



\subsection{Reguläre konvexe Polyeder in }


\subsection{Fibonacci Gitter auf $\mathrm{SU}(2)$}







Zunächst betrachten wir Partitionierungen, der $\mathrm{SU}(2)$. Ein allgemeines Element $U \in \mathrm{SU}(2)$ in der fundamentalen Representation kann geschrieben werden als
\begin{align*}
U &=
\begin{pmatrix}
  u & w \\
  -\overline{w} & \overline{u}
\end{pmatrix} \qquad \textrm{mit} \quad u,w \in \mathbb{C} \quad \textrm{und} \quad \abs{u} + \abs{w} = 1
\end{align*}
Hier zeigt sich sofort, dass $\mathrm{SU}(2)$ diffeomorph zu $S^3$ ist, der Einheitssphäre im $\mathbb{R}^4$.

\subsection{Quellen}

https://journals.aps.org/prd/pdf/10.1103/PhysRevD.27.412
https://journals.aps.org/prd/pdf/10.1103/PhysRevD.22.2465


\begin{itemize}
\item Naive Partitionierung von $S_2$ und $S_3$
\item Ikosaeder in 4 Dimensionen und abgefahrenes Zeug was man damit anstellen kann $H_4$ (\url{https://arxiv.org/pdf/1705.04910.pdf})
\item Fibonacci Latice (\url{https://stackoverflow.com/questions/57123194/how-to-distribute-points-evenly-on-the-surface-of-hyperspheres-in-higher-dimensi} and \url{https://math.stackexchange.com/questions/3291489/can-the-fibonacci-lattice-be-extended-to-dimensions-higher-than-3/3297830#3297830})

SO(3) Untergruppen: http://math.uchicago.edu/~may/REU2020/REUPapers/Bui,An.pdf
