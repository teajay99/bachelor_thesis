A general Element $U \in \mathrm{SU}(2)$ in the fundamental representation can be written as
\begin{align*}
 U & =
 \begin{pmatrix}
  u             & w            \\
  -\overline{w} & \overline{u}
 \end{pmatrix} \qquad \textrm{mit} \quad u,w \in \mathbb{C} \quad \textrm{und} \quad \abs{u} + \abs{w} = 1
\end{align*}
\cite{findsource}. Somit ist $\mathrm{SU}(2)$ diffeomorph zu $S^3$, der Einheitssphäre im $\mathbb{R}^4$, und equivalent zur Gruppe der Einheitsquaternionen
\begin{align*}
 \mathbb{H}_1 = \left\{ x \in \mathbb{H} : \norm{x} = 1\right\} \textrm{.}
\end{align*}
\cite{findsource}. Zudem ist $\mathrm{SU}(2)$ die zweifache Überlagerung der Drehgruppe $\mathrm{SO}(3)$ in $\mathbb{R}^3$ \cite{findsource}. Im folgenden sollen diese Eigenschaften genutzt werden, um representative endliche Untermengen von $\mathrm{SU}(2)$ zu finden.

\subsection{Finite Subgroups of \SUTwo}

The first thing to take a look at, in terms of representative finite subsets of \SUTwo are its finite subgroups. As first proven in \cite{findsource} these can be constructed by taking the cartesian product of $\mathbb{Z}_2$ with the subgroups of $\mathrm{SO}(3)$. The finite subgroups of $\mathrm{SO}(3)$ can be constructed by all the symmetry transformations of regular polygons as well as rotations of the platonic solids \cite{findsource}. The groups generated by the regular polygons where however not considered in this work, as they are restricted to planes. Therefore it is unlikely that their elements will give a representative subset of \SUTwo.\\

The five platonic solids give rise to three distinct subgroups. The first one would be the tetrahedral group $T$ with $12$ elements, which as the name suggest is generated by the orientation preserving rotations of a Tetrahedron. Next up is the octahedral group $O$ with $24$ elements, generated by the rotational symmetries of the Octahedron ( or the Cube respectively as it is the geometric dual of the Octahedron). Finally there is the icoshedral group $I$ with $60$ elements generated by the rotational symmetries the Icosahedron and Dodecahedron.

Expanding these groups to \SUTwo doubles the element counts and therefore gives rise to the so called binary tetrahedral group $\overline{T}$, the binary octahedral group $\overline{O}$ and the binary icoshedral group $\overline{I}$, with $24$, $48$ and $120$ elements respectively. One possible representation can be found in table \ref{tab:subgroups}.

\begin{table}
 \centering
 \begin{tabular}{c|c|l}
  group          & order & elements                                                                                                                                              \\
  \hline
  $\overline{T}$ & 24    & all sign combinations of $\left\{\pm 1, \pm i, \pm j, \pm k, \frac{1}{2}\left( \pm 1 \pm i \pm j \pm k \right) \right\} $                             \\
  \hline
  $\overline{O}$ & 48    & all sign combinations and permutations of                                                                                                             \\
                 &       & $\left\{\pm 1, \pm i, \pm j, \pm k, \frac{1}{2}\left( \pm 1 \pm i \pm j \pm k \right) , \frac{1}{\sqrt{2}}\left( \pm 1 \pm i \right) \right\} $       \\
  \hline
  $\overline{I}$ & 120   & all sign combinations and even permutations of of                                                                                                     \\
                 &       & $\left\{\pm 1, \pm i, \pm j, \pm k, \frac{1}{2}\left( \pm 1 \pm i \pm j \pm k \right) , \frac{1}{2}\left(1+\tau i + \frac{j}{\tau} \right) \right\} $ \\
 \end{tabular}
 \caption{Quaternionic Representation of $\overline{T}$, $\overline{O}$ and $\overline{I}$ as found in \cite{duval:1964}, where $\tau= \frac{1+\sqrt{5}}{\sqrt{2}}$ denotes the golden ratio}
 \label{tab:subgroups}
\end{table}

\subsection{Regular Polytopes in four dimensions}

The other nice thing about the platonic solids, is that their vertices are, if scaled accordingly, a well distributed set of points on $S_2$. As \SUTwo is diffeomorphic to $S_3$ a natural next step would therefore be to take a look at the regular polytopes in four dimensions. In total there are six of those denoted by the number of cells (i.e. three dimensional faces) they have. The vertex count ranges from the 5 vertices of the 5-cell $C_5$ to the 600 vertices of the 120-cell $C_{120}$. The vertices of all six can be found in table \ref{tab:polytopes}.

It is noteworthy that the vertices of $C_{24}$ and  $C_{600}$ coincide with $\overline{T}$ and $\overline{I}$ respectively. The vertices of $C_{16}$ also form a group called the binary dihedral group $\overline{D}_4$ \cite{findsource}.

\begin{table}
 \centering
 \begin{tabular}{c|c|l}
  polytope  & \# vertices & vertices                                                                                                                                    \\
  \hline
  $C_5$     & 5           & $\begin{array}{l}
    \left\{ 1,  -\frac{1}{4} + \eta i + \eta j + \eta k,  -\frac{1}{4} + \eta i + \eta j  + \eta k, \right. \\
    \left.  -\frac{1}{4} + \eta i + \eta j + \eta k, -\frac{1}{4} + \eta i + \eta j + \eta k \right\}
   \end{array}$                                                                                                                 \\
  \hline
  $C_{16}$  & 8           & $\left\{ \pm 1, \pm i, \pm j, \pm k \right\}$                                                                                               \\
  \hline
  $C_8$     & 16          & all sign combinations of $ \left\{  \frac{1}{2} (\pm 1 \pm i \pm j \pm k)\right\}$                                                          \\
  \hline
  $C_{24}$  & 24          & $C_8 \cup C_{16}$ (same as $\overline{T}$)                                                                                                  \\
  \hline
  $C_{600}$ & 120         & \makecell[l]{ $ C_{24} \cup \left\{  \textrm{sign comb. and even perm. of } \frac{1}{2} \left(1+\tau i + \frac{j}{\tau} \right)  \right\} $ \\ (same as $\overline{I}$)} \\
  \hline
  $C_{120}$ & 600         & $\{a b \, | \, a \in C_5, \, b \in C_{600} \}$                                                                                              \\
 \end{tabular}
 \caption{title}
 \label{tab:polytopes}
\end{table}

\subsection{Geodesic Polytopes in Four Dimensions}

Another nice property of the platonic solids, is that there faces are regular polygons. This allows for the construction of so called geodesic polyhedra \ref{findsource}. As seen in figure \ref{} the basic idea here is to subdivide the faces of an platonic solid into triangles and then project the obtained vertices onto the sphere. Although the cells of the regular polytopes in four dimensions are the platonic solids, such a procedure is complicated by the fact that most of the platonic solids do not tile space on their own. Therefore finding appropriate subdivision in a scaleable way is not a trivial task.\\

The only exception to this would be $C_8$, as its eight cells are cubes. In this case any of the eight \emph{surface cubes} can be trivially filled with $m^3$ cubes of sidelength $\frac{1}{m}$ for $m \in \mathbb{N}/\{0\}$. As seen in figure \ref{} the grid obtained by this procedure in three dimensions is somewhat reminiscent of a volleyball. So in the following we will denote such a \emph{volleyball lattice}  with n subdivision by $V_n$. \\

As any of the eight surface cubes is fixed by holding one of the coordinates at $\pm 1$ the set of coordinates before projection onto the sphere $\tilde{V}_n$ is obtained by
\begin{align*}
 \tilde{V}_n & = \left\{ \textrm{all permutations of }  \, \frac{1}{2} \left( \pm 1 + v_n(a) i + v_n(b) j + v_n(c) k \right)  \, \middle| \, a,b,c \in \mathbb{N}: 0 \le a,b,c \le n+1 \right\}
\end{align*}
with $v_n$ beeing defined as
\begin{align*}
 \quad v_n(m)  =  1-\frac{2m}{n+1} \, \textrm{.}
\end{align*}
$V_n$ is then simply obtained by dividing out the norm for every point in $\tilde{V}_n$:
\begin{align*}
 V_n = \left\{\frac{q}{\abs{q}} \, \middle| \,  q \in \tilde{V}_n \right\} \, \textrm{.}
\end{align*}
The number of distinct elements of $V_n$ can be calculated to be
\begin{align*}
 \abs{V_n} & = \underbrace{16}_{\substack{ \textrm{vertices of } C_8
 }} + \underbrace{8 n^3}_{\substack{n^3 \, \textrm{vertices added in}             \\ \textrm{every one of the 8 cells}}} + \underbrace{24 n^2}_{\substack{n^2 \, \textrm{vertices added on} \\ \textrm{every one of the 24 faces}}} + \underbrace{32 n}_{\substack{n \, \textrm{vertices added on} \\ \textrm{every one of the 32 edges}}} \, \textrm{.}
\end{align*}.

The main advantage of the \emph{volleyball} lattice is that it scales, and therefore allows some control over the granularity of the gauge set. Due to the method of construction the density of vertices will however increase towards the vertices of $C_8$. This will probably lead to some systematic deviations when using $V_n$ as an approximation of \SUTwo.

\subsection{Fibonacci lattice on \SUTwo}

The final discretization of \SUTwo considered in this work is the so called Fibonacci lattice. \ref{stack:fibonacci}

%\begin{align*}
%
%\end{align*}

The fibonacci approach offers two important advantages, compared to the previously proposed discretizations. First of all it can generate a representative gauge set for any given number of elements. As the motivation for these discretizations is to be used for simulations on quantum computers, this effectively means that one will always be able to achieve maximum granularity for a given amount of quantum memory \footnote{It should be noted, that the algorithms for such simulations of \SUTwo gauge theorys are still in the works, and therefore way beyond the scope this work. The stated advantage is therefore based on a reasonable assumption and not a know fact.}.\\

The way the four dimensional version was constructed also suggest, that one can obtain similar procedures for other compact manifolds and therefore other continuous gauge groups. To put this statement into rigorous mathematical terminology or to provide a proof is beyond the mathematical abilities of the author.

However as long as one manages to introduce coordinates $(y_1, ..., y_n)$ on the manifold, and the metric $g$, obtained by a pullback from $n+1$ dimensional euclidian space, takes a form s.t. $\abs{g} = \Pi_{i} \, f_i(y_i)$, one should be able to obtain a set of equations similar to \ref{}, of which the solutions should have similiar properties. Therefore there is at least a good chance that such lattices can e.g. be constructed for $\textrm{SU}(3)$ and therefore be used for QCD simulations.

\subsection{Quellen}

https://journals.aps.org/prd/pdf/10.1103/PhysRevD.27.412

https://journals.aps.org/prd/pdf/10.1103/PhysRevD.22.2465


\begin{itemize}
 \item Naive Partitionierung von $S_2$ und $S_3$
 \item Ikosaeder in 4 Dimensionen und abgefahrenes Zeug was man damit anstellen kann $H_4$ (\url{https://arxiv.org/pdf/1705.04910.pdf})
 \item Fibonacci Latice (\url{https://stackoverflow.com/questions/57123194/how-to-distribute-points-evenly-on-the-surface-of-hyperspheres-in-higher-dimensi} and \url{https://math.stackexchange.com/questions/3291489/can-the-fibonacci-lattice-be-extended-to-dimensions-higher-than-3/3297830#3297830})
\end{itemize}
SO(3) Untergruppen: http://math.uchicago.edu/~may/REU2020/REUPapers/Bui,An.pdf
