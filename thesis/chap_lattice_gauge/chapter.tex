The action of a pure gauge theory of gauge group $G$ is given by
\begin{align*}
 S_G & = -\frac{1}{4} \int \mathrm{d}^4 x \, \mathrm{Tr} \left[ F_{\mu \nu} (x) F^{\mu \nu} (x) \right]
\end{align*}
where the matrices $F_{\mu \nu}$ are given by
\begin{align*}
 F_{\mu \nu} & =  \partial_\mu A_\nu - \partial_\nu A_\mu  -  i g [A_\mu, A_\nu] \textrm{.}
\end{align*}
with gauge field $A_\mu = A_\mu^a t_a$ in terms of the generators $t_a$ of $G$ as well as the coupling constant $g$ \cite{Peskin:1995}. In order to treat such a theory numerically one first transitions to imaginary time. This is done by $x^0 \rightarrow -i x_4$ which leads to $S_G \rightarrow i S_G^{(\textrm{eucl.})}$ with
\begin{align*}
 S_G^{(\textrm{eucl.})} & = \frac{1}{4} \int \mathrm{d}^4 x  \sum_{\mu,\nu}  \, \mathrm{Tr} \left[ F_{\mu \nu} (x) F_{\mu \nu} (x) \right] \textrm{.}
\end{align*}
where now $\eta_{\mu \nu} = \delta_{\mu \nu}$ and therefore $F_{\mu \nu} = F^{\mu \nu}$. Furthermore one introduces a four dimensional space-time lattice  with $N^4$ lattice sites, periodic boundary conditions and lattice spacing $a$. This is achieved by the following replacements:
\begin{align*}
 x^\mu               & \rightarrow a \, n^\mu                           \\
 \int \mathrm{d}^4 x & \rightarrow a^4 \sum_{n \in \mathbb{N}^4_N}        \\
 A_\mu (x)           & \rightarrow A_\mu(an) \equiv A_\mu(n) \textrm{.}
\end{align*}
where
\begin{align*}
 \mathbb{N}^4_N \in \left\{ (a,b,c,d)^T \in \mathbb{N}^4 \, \middle| \, 0 \le a,b,c,d < N \right\} \textrm{.}
\end{align*}
In order to treat the interaction of the gauge fields with additional fermion or scalar fields it is also typical to introduce the so called link variables $U_\mu$ given by
\begin{align*}
 U_\mu(n) = \exp \left( i g a A_\mu(n) \right) \in G \textrm{.}
\end{align*}
With this in place one can introduce the plaquette product $U_{\mu \nu}$:
\begin{align*}
 U_{\mu \nu}(n) & = U_\mu(n) \, U_\nu(n+e_\mu) \, U^\dagger_\mu (n + e_\nu) U^\dagger_\nu(n)
\end{align*}
where the term plaquette denotes the path ordered product around a loop on the lattice as shown in figure \ref{}. Using the Baker-Campbell-Hausdorf formula this can be calculated to be
\begin{align*}
 U_{\mu \nu}(n) & = \exp \left( iga^2 F_{\mu \nu} (n) \right) + \mathcal{O}(a^5) = \mathbb{1} + i g a^2 F_{\mu \nu} (n) - \frac{g^2 a^4}{2} (F_{\mu \nu} (n))^2 + \mathcal{O}(a^5)
\end{align*}
and thus
\begin{align*}
 U_{\mu \nu}^\dagger(n) & = \exp \left( - iga^2 F_{\mu \nu} (n) \right) + \mathcal{O}(a^5) = \mathbb{1} - i g a^2 F_{\mu \nu} (n) - \frac{g^2 a^4}{2} (F_{\mu \nu} (n) )^2 + \mathcal{O}(a^5) \textrm{.}
\end{align*}
Such simplifications are reasonable as the physical continuum limit is obtained by taking the lattice spacing $a \rightarrow 0$. Terms up to order four in $a$ need to be considered, as they are required to transition from the sum over $n$ to the original integral again. Higher order terms however will just vanish when taking the continuum limit and can therefore be dropped.

With this we can now express $F_{\mu \nu}$ in terms of $U_{\mu \nu}$:
\begin{align*}
 a^4 (F_{\mu \nu})^2 & \approx \frac{2}{g^2} \left( \mathbb{1} - \frac{1}{2}\left( U_{\mu \nu} + U_{\mu \nu}^\dagger \right) \right)
\end{align*}
which leaves us with the euclidean lattice action
\begin{align*}
 S^{(\textrm{lat.})}_G & = \frac{1}{4} \sum_{n \in \mathbb{N}^4_N} \sum_{\mu,\nu} a^4 \, \mathrm{Tr} \left[ \left( F_{\mu \nu} (n) \right)^2  \right]                                                        \\
                       & = \frac{1}{2} \sum_{n \in \mathbb{N}^4_N} \sum_{\mu < \nu}  a^4 \, \mathrm{Tr} \left[ \left( F_{\mu \nu} (n) \right)^2  \right]                                                     \\
                       & \approx \frac{\beta}{2} \sum_{n \in \mathbb{N}^4_N} \sum_{\mu < \nu} \, \mathrm{Tr} \left[ \mathbb{1} - \frac{1}{2}\left( U_{\mu \nu} (n) + U_{\mu \nu}^\dagger (n) \right) \right]
\end{align*}
where $\beta = \frac{2}{g^2}$. For $G=\mathrm{SU}(2)$ the generators $t$ will be $2 \cross 2$ matrices given by $t_a = \frac{\sigma_a}{2}$ where $\sigma_a$ denotes the three pauli matrices. This means that $\mathrm{Tr}\left[ \mathbb{1} \right] = 2$. As can be seen in the explicit parameterization given later in eq.  \ref{eq:generalSu2Element} also $ \mathrm{Tr} \left[ U_\mu \right] = \mathrm{Tr} \left[ U_\mu^\dagger \right] $ holds for elements of \SUTwo. This allows one to simplify the action to be
\begin{align*}
 S^{(\textrm{lat.})}_{\mathrm{SU}(2)} & = \beta \sum_{n \in \mathbb{N}^4_N} \sum_{\mu < \nu} \left( 1 - \frac{1}{2} \mathrm{Tr} \left[ U_{\mu \nu} (n) \right] \right) \textrm{.}
\end{align*}
This will be the action used in the rest of this work and consequently referred to simply as $S_G$.
% Lastly it is typical to
% \begin{align*}
%  S_G^{(\textrm{eucl.})} & = \beta \sum_{P \in \Lambda} \left(1-\frac{1}{2} \textrm{Tr} \left[ U_P \right] \right)
% \end{align*}
This derivation was taken from \ref{rothe:2005} where it can be found in more detail and generality.
