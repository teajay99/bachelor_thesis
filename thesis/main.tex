%!TeX spellcheck = en-DE

\documentclass[a4paper]{scrartcl}
\usepackage{packages}


\title{Bachelor Thesis}
\author{Timo Jakobs \\\\ \textbf{University of Bonn}}
\date{\today}

\newcommand{\SUTwo}{$\mathrm{SU}(2)$\xspace}

\tabulinesep=1.2mm

\begin{document}
\maketitle

\begin{abstract}
In this work we will take a look at different approaches to approximating the group \SUTwo in its function as the gauge group of a quantum field theory. This includes finite subgroups of \SUTwo like the popular icosahedral approximation, four-dimensional versions of the platonic solids and geodesic polytopes, as well as a generalized fibonacci lattice. In order to test these approximations Metropolis Monte Carlo simulations were implemented and run.
\end{abstract}

\tableofcontents
\newpage




\section{The lattice action}
\import{chap_introduction/}{chapter.tex}

\section{Discretisations of \SUTwo}
\label{sec:discSu2}
\import{chap_discrete_su_2/}{chapter.tex}

\section{Metropolis Monte Carlo Method}
\import{chap_monte_carlo/}{chapter.tex}

\section{Results}
\label{sec:results}
\import{chap_results/}{chapter.tex}

\section{Conclusion}
\label{sec:conclusion}
\import{chap_conclusion/}{chapter.tex}

\printbibliography

\begin{appendix}
\section{Recorded Data}
%\import{appendix_raw_data/}{raw_data.tex}
\end{appendix}

\end{document}
