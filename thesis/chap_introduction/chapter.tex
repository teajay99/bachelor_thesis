The action of a pure gauge theory of gauge group $G$ is given by

\begin{align*}
 S_G                                   & = -\frac{1}{4} \int \mathrm{d}^4 x  \sum_i F_{\mu \nu}^{i} (x) F^{i \, \, \mu \nu} (x) \\
 \textrm{with} \qquad  F_{\mu \nu}^{a} & =  \partial_\mu A^{a}_\nu - \partial_\nu A^{a}_\mu  + g f^{abc} A^{b}_\mu A^{c}_\nu
\end{align*}
with gauge field $A_\mu^a$, coupling constant $g$ and structure constants $f^{abc}$\cite{Peskin:1995}. In order to treat such theory numerically one first transitions to imaginary time. This is done by $x^0 \rightarrow -i x_4$ which leads to $S_G \rightarrow i S_G^{(\textrm{eucl.})}$ with

\begin{align*}
 S_G^{(\textrm{eucl.})} & = \frac{1}{4} \int \mathrm{d}^4 x  \sum_{i,\mu,\nu} F_{\mu \nu}^{(i)} (x) F_{\mu \nu}^{(i)} (x)  \textrm{.} \\
\end{align*}

Furthermore one introduces a four dimensional space-time lattice with periodic boundary conditions and lattice spacing $a$. This is achieved by the following replacements:

\begin{align*}
 x^\mu               & \rightarrow a \, n^\mu                \\
 \int \mathrm{d}^4 x & \rightarrow a^4 \sum_n                \\
 A_\mu (x)           & \rightarrow A_\mu(an) \equiv A_\mu(n) \textrm{.}
\end{align*}

\begin{align*}
 U_\mu(n) = \exp \left( i g a A_\mu(n) \right) \in G
\end{align*}

\begin{align*}
 U_{\mu \nu}(n) & = U_\mu(n) \, U_\nu(n+e_\mu) \, U^\dagger_\mu (n + e_\nu) U^\dagger_\nu(n) \\
                & = ...                                                                      \\
                & = \exp \left( iga^2 F_{\mu \nu} \right) + \mathcal{O}(a^3)
\end{align*}




\begin{align*}
 U_{\mu \nu} &= 1 + i g a^2 F_{\mu \nu} + \mathcal{O}(a^3) \\
 \Leftrightarrow \qquad F_{\mu \nu} &=
\end{align*}

The generators of \SUTwo can be written as $\frac{\sigma^i}{2}$ in terms of the pauli matrices $\sigma^i$. As
\begin{align*}
 \left[ \frac{\sigma^a}{2}, \frac{\sigma^b}{2} \right] & = \frac{i}{2} \varepsilon^{abc} \, \sigma^c \textrm{.}
\end{align*}
holds, the structure constants are given by $f^{abc} = \varepsilon^{abc}$.



\begin{align*}
 S_G^{(\textrm{eucl.})} & = \beta \sum_{P \in \Lambda} \left(1-\frac{1}{2} \textrm{Tr} \left[ U_P \right] \right)
\end{align*}
