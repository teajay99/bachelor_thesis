Finally, it is time to code up the simulations. This was done in \texttt{C++} making use of Nvidia's \texttt{CUDA} library to allow for parallel execution on GPUs. The source code including further documentation can be found at \url{https://github.com/teajay99/bachelor_thesis}.\\

\subsection{Validation of the Continuous Case}

First up, we need to figure out how to choose the $\delta$ parameter when generating the transition elements $V$ in eq. \ref{eq:transitionV}. This is a bit of a balancing act, as for bigger values of $\delta$ fewer configuration changes will be accepted, leading to successive configurations being very similar. With small $\delta$, more of the configuration changes will be accepted, but now these changes are very small and the successive configuration will still look a lot like the previous one.

The good news is however that the choice of $\delta$ is not critical, as for a sufficient amount of iterations, it will not affect the measured quantities themselves. Correlation between successive configurations however need to be taken into account when estimating the uncertainties of a measured quantity. This was done using the \texttt{uwerrprimary} function from the \emph{R} library \texttt{hadron}.\\

To evaluate how well $\delta$ is chosen, one can take a look at the acceptance rate $\frac{N_{\textrm{acc}}}{N_{\textrm{hit}}}$, i.e. the ratio of line \ref{code:metroMonteIfCond} in algorithm \ref{alg:metroMonte} evaluating to \texttt{true} and the total number of evaluations. A good rule of thumb is to get the acceptance rate to about $\SI{50}{\%}$.
\begin{figure}[!hbt]
 \centering
 \begin{subfigure}[t]{0.49\textwidth}
  \input{calibHitRate.pgf}
  \caption{Acceptance Rate Calibration}
  \label{plot:calibHitRate}
 \end{subfigure}
 \begin{subfigure}[t]{0.49\textwidth}
  \input{referenceDataSet.pgf}
  \caption{$P(\beta)$ measured on an $8^4$ lattice for $100,000$ iterations. All error bars are significantly smaller than the plotted points and where therefore omitted.}
  \label{plot:refDataSet}
 \end{subfigure}
 \caption{}
\end{figure}

In order to reliably meet this rule, a simply zero search for the function
\begin{align*}
 f(\delta) & = \left(\frac{N_{\textrm{acc}}}{N_{\textrm{hit}}}\right)(\delta, \beta) - 0.5
\end{align*}
was implemented and run for 25 logarithmically spaced values of $\beta$ between $\beta=0.1$ and $\beta = 100$. The results with their corresponding acceptance rate can be found in figure \ref{plot:calibHitRate}. Also, a function of the form
\begin{align*}
 \delta(\beta) = \min \left(1,\frac{a}{\beta-e} + \frac{b}{(\beta-e)^2} +  \frac{c}{(\beta-e)^3} + \frac{d}{(\beta-e)^4} + f \right)
\end{align*}
was fitted to the data, with the results reading
\begin{align*}
 \begin{array}{rlrl}
  a= & 1.6477  \qquad \qquad & b= & -6.9405               \\
  c= & 18.415  \qquad \qquad & d= & -0.03858 \\
  e= & -1.3638 \qquad \qquad & f= & 0.02372 \textrm{.}    \\
 \end{array}
\end{align*}
The fit looks reasonable, and will therefore be used to serve the values for $\delta$ in the following measurements. A further inspection into e.g. the uncertainties of the fit parameters or error of regression were not done, as again the exact choice of $\delta$ is not critical, and the fit itself has no physical implications.

Another trick typically implemented to decrease the correlation between successive configurations is to probe each link variable multiple times before moving on to the next one. For all the simulations here, this was done $10$ times.\\

The only thing left is to pick is an operator $\mathcal{O}$ to measure. A simple and popular choice here is to go with the average plaquette $P$ measured by
\begin{align*}
 P = \frac{1}{\mathcal{N}_P} \sum_{n \in \mathbb{N}^4_N} \sum_{\mu < \nu} \frac{\beta}{2} \mathrm{Tr} \left[ U_{\mu \nu} (n) \right]
\end{align*}
where $\mathcal{N}_P = 6 N^4$ denotes the number of different plaquettes for a lattice of size $N^4$.

With this we then ran the continuous case for $\beta \in \{0.1,0.2,\dots,9.8,9.9,10\}$ on an $8^4$ lattice for $104,000$ iterations starting from a hot configuration. The first $4,000$ iterations were treated as thermalization iterations, and the successive $100,000$ will be used as reference data set, when evaluating the approximations of \SUTwo. The obtained values are plotted in figure \ref{plot:refDataSet}.

\begin{figure}[!hbt]
 \begin{subfigure}[t]{0.49\textwidth}
  \input{calibStrongCoupling.pgf}
  \caption{Strong coupling expansion (red) compared to the measured data set (black). The uncertainties of the measured data are significantly smaller than the plotted points.}
  \label{plot:strongCoupling}
 \end{subfigure}
 \begin{subfigure}[t]{0.49\textwidth}
  \input{calibWeakCoupling.pgf}
  \caption{Weak coupling expansion (red) compared to the measured data set (black). The uncertainties of the measured data are significantly smaller than the plotted points.}
  \label{plot:weakCoupling}
 \end{subfigure}
 \caption{}
 \centering
\end{figure}

In order to verify that the measured data is sound, one can take a look at the so-called strong coupling expansion given by
\begin{align*}
 P_{\textrm{s. c.}}(\beta) = \frac{1}{4} \beta - \frac{1}{96} \beta^3 + \frac{7}{1536} \beta^5 - \frac{31}{23040} & \beta^7 + \frac{4451}{8847360} \beta^9 - \frac{264883}{1486356480} \beta^{11}       \\
 +                                                                                                                & \frac{403651}{5945425920} \beta^{13} - \frac{1826017873}{68491306598400} \beta^{15}
\end{align*} as found in \cite{Denbleyker:2008}. As the name suggest, this approximates $P$ for big values of the coupling constant $g$, and therefore small values of $\beta$. As can be seen in figure \ref{plot:strongCoupling}, the measured values closely agree with the prediction up to about $\beta = 1.2$.

The same article also gives a weak coupling expansion in $\frac{1}{\beta}$ up to order fifteen for lattice {size $6^4$}:
\begin{align*}
 P(\beta)_{\textrm{w. c.}} & = \sum_{i=1}^{15} \frac{\gamma_i}{(\beta)^i} \textrm{,}
\end{align*}
where the coefficients $\gamma_i$ can be found in table \ref{tab:weakCoupleCoff}. For this another data set with $10,000$ iterations was generated to account for the smaller lattice size. The results can be found in figure \ref{plot:weakCoupling}. The values agree within the error bars up to about $1/\beta = 0.25$, and then start to slowly deviate. Nevertheless, they confirm that the simulation is working correctly.\\
\begin{table}[!hbt]
 \centering
 \begin{tabular}{L || L | L | L | L | L | L | L | L}
  i        & 1      & 2      & 3      & 4      & 5      & 6     & 7      & 8    \\
  \hline
  \gamma_i & 0.7498 & 0.1511 & 0.1427 & 0.1747 & 0.2435 & 0.368 & 0.5884 & 0.98 \\
 \end{tabular}\\
 \vspace{5mm}
 \begin{tabular}{L || L | L | L | L | L | L | L }
  i        & 9      & 10     & 11    & 12     & 13     & 14     & 15     \\
  \hline
  \gamma_i & 1.6839 & 2.9652 & 5.326 & 9.7234 & 17.995 & 33.690 & 63.702 \\
 \end{tabular}
 \caption{Weak coupling expansion for a $6^4$ lattice}
 \label{tab:weakCoupleCoff}
\end{table}

As we now have a solid data set to compare against, we can take a look at the proposed approximations of \SUTwo. This will be done in two steps. As pointed out in \cite{Petcher:1980}, finite gauge groups usually show a phase transition towards higher values of $\beta$. Finding these will therefore be our first concern. In the second step, we will also take a look at systematic deviations for some specific values of $\beta$.

\subsection{Phase Transitions}

For each of the considered gauge sets, $4000$ thermalization iterations were run, and another $3000$ for measurements afterwards. This was repeated twice, once with a hot start, and once with a cold one. The results are plotted in figures \ref{plot:subgroups} to \ref{plot:fibonacciII}. Shown are the deviations $P - P_{\textrm{ref}}$ from the previously generated reference data. The y-axis was set to be linear for the interval $[-10^{-3},10^{-3}]$ and logarithmic beyond that, in order to display both the statistical fluctuations, as well as systematic deviations in a meaningful way.\\
\begin{figure}[!hbt]
 \centering
 \input{Subgroups.pgf}
 \caption{Scan in $\beta$ for the finite subgroups of \SUTwo}
 \label{plot:subgroups}
\end{figure}

We begin with the finite subgroups $\overline{T}$, $\overline{O}$ and $\overline{I}$ for which the results can be found in figure \ref{plot:subgroups}. Here one can see that for small $\beta$, the measured values mostly agree with the continuous case. Towards bigger values of $\beta$, some systematic deviations show up, until the lattice configuration undergoes a phase transition and \emph{freezes} at $P\approx1$. To measure $P=1$ all link variables need to be at the same gauge set element. This happens, as  $\Delta S$, because of the discretization, now has a minimum value. This would be given by the difference in action of a cold lattice, and one where one of the link variables transitioned to an adjacent vertex. For a given $\beta_{\textrm{ph.}}$, $\Delta S$ then reaches a critical value, such almost all fluctuations will be rejected.

The continuous case does not show this behavior, as for any $\beta$ configuration changes with a sufficiently small $\Delta S$ exist. This therefore always allows for some small fluctuations, decreasing the measured value of $P$. Thus, all the proposed approximations will only be useful for values where $\beta < \beta_{\textrm{ph.}}$.

Another interesting feature of figure \ref{plot:subgroups} is, that the cold and hot starting configurations form a hysteresis loop around the phase transition. This indicates, that around $\beta_{\textrm{ph.}}$ multiple semi stable configurations exist. It can be expected that this can be partially fixed, by applying more thermalization iterations. Further, testing on $\overline{I}$ however showed, that these loops do not entirely disappear, even for significantly higher numbers of iterations.

In \cite{Petcher:1980} the exact positions of the phase transitions were found to be $\beta_{\textrm{ph.}}(\overline{T}) = 2.15 \pm 0.05$, $\beta_{\textrm{ph.}}(\overline{O}) = 3.25 \pm 0.01$ and $\beta_{\textrm{ph.}}(\overline{I}) = 6.01 \pm 0.01$.
Estimating these transitions, by taking the center of the corresponding hysteresis loop as $\beta_{\textrm{ph.}}$ and the width of the loop as two standard deviations we can read of $\beta_{\textrm{ph.}}(\overline{T}) = 2.15 \pm 0.15$, $\beta_{\textrm{ph.}}(\overline{O}) = 3.2 \pm 0.1$ and $\beta_{\textrm{ph.}}(\overline{I}) = 5.7 \pm 0.2$. Thus, the values for $\overline{T}$ and $\overline{O}$ agree with \cite{Petcher:1980}, while $\beta_{\textrm{ph.}}(\overline{I})$ was measured slightly lower. The reason for this difference is most likely the limited computing power of the time.\\
\begin{figure}[!hbt]
 \centering
 \input{RegularPolytopes.pgf}
 \caption{Scan in $\beta$ for the remaining regular polytopes}
 \label{plot:regPolytopes}
\end{figure}

The remaining regular polytopes $C_5$, $C_{16}$, $C_8$ and $C_{120}$ for which the results can be found in figure \ref{plot:regPolytopes} give a similar picture. As maybe expected due to the small vertex count, $C_{5}$, $C_{16}$ and $C_{8}$ performed not that well with the phase transition as low as $\beta_{\textrm{ph.}}(C_5) = 1.2 \pm 0.2$, $\beta_{\textrm{ph.}}(C_{16}) = 1.15 \pm 0.15$ and $\beta_{\textrm{ph.}}(C_{8}) = 1.9 \pm 0.2$.
$C_{120}$ with it's 600 vertices however pretty much matches \SUTwo for the tested range of $\beta$, with only slight systematic deviations towards the higher values of $\beta$, and no phase transition found.

Another thing to note here is, that beyond the phase transition the values of the hot and cold starting configuration start to diverge again. This is most likely due to an insufficient amount of thermalization iterations for the hot lattice to \emph{freeze}. As we are however not too concerned with that region anyway and the hot and cold starting configuration agree to the left of the phase transition, this is not a problem.\\

\begin{figure}[!hbt]
 \centering
 \input{Volleyball.pgf}
 \caption{Scan in $\beta$ for Volleyball lattices}
\end{figure}
Of the volleyball lattices, only $V_1$ showed a phase transition at $\beta_{\textrm{ph.}}(V_1) = 4.95 \pm 0.05 $. This makes sense as it only has $80$ vertices, while $V_2$ already has $240$, $ V_3 $ has $544$ and $ V_4 $ has $1040$. For $V_2$, $V_3$ and $V_4$ the expected systematic deviations show up prominently for $\beta > 2$. For $V_3$ and $V_4$ they seem to be consistently around $5 \cdot 10^{-4}$ with a slight decrease towards higher values of $\beta$. The increase in deviation for $V_2$ towards bigger values of $\beta$ is most likely an indication for an upcoming phase transition.\\

\begin{figure}[!hbt]
 \centering
 \input{Fibonacci-I.pgf}
 \caption{Scan in $\beta$ for Fibonacci lattices on \SUTwo (Part I)}
 \label{plot:fibonacciI}
\end{figure}
\begin{figure}[!hbt]
 \centering
 \input{Fibonacci-II.pgf}
 \caption{Scan in $\beta$ for Fibonacci lattices on \SUTwo (Part II)}
 \label{plot:fibonacciII}
\end{figure}

Lastly, we take a look at the Fibonacci lattices. These were tested for lattice sizes  $8$, $16$, $32$, $64$, $128$, $256$ and $512$. The results can be found in figures \ref{plot:fibonacciI} and \ref{plot:fibonacciII}. Reading of the positions of the phase transitions in the same way as before is not really possible here, as for lattice sizes bigger than $16$ the hot configurations didn't really freeze. This is most likely due to the inefficient generation of new random elements. Nevertheless, we can put a lower bound on $\beta_{\textrm{ph.}}$ by taking a look at where the cold starting configuration transitions. With this we get $\beta_{\textrm{ph.}}(F_8) = 1.55 \pm 0.25$, $\beta_{\textrm{ph.}}(F_16) = 2.55 \pm 0.35 $, $\beta_{\textrm{ph.}}(F_{32}) > 3.5 $, $\beta_{\textrm{ph.}}(F_{64}) > 5.2 $ and $\beta_{\textrm{ph.}}(F_{128}) > 7.8$. $F_{256}$ and $F_{512}$ reproduced the reference data set for the whole tested range of $\beta$, with only slight systematic deviations for $F_{256}$ towards higher $\beta$. This is probably indicating an approaching phase transition beyond the measured range.

Surprisingly the Fibonacci lattices seem to perform significantly better than e.g. the subgroups in terms of phase transitions. $F_{32}$ shows systematic deviations at around $\beta = 2$, similar to $\overline{O}$. The actual phase transition of $F_{32}$ however happens for a slightly higher value of $\beta$, although it only has two thirds of the vertices. In the same way, $\overline{I}$ is significantly outperformed by $F_{128}$ with only eight vertices more. \\

\begin{figure}[!hbt]
 \centering
 \input{fibPhaseScan.pgf}
 \caption{Phase transitions for different sized Fibonacci lattices, compared to the previous approximations of \SUTwo.}
 \label{plot:fibPhaseScan}
\end{figure}

To get a better idea of how $\beta_{\textrm{ph.}}$ scales with the lattice size $n$ for the Fibonacci lattice, we ran another scan for $40$ logarithmically spaced lattice sizes between $n=8$ and $n=256$. For this $\beta$ was increased in steps of $0.1$ until $P-P_{\textrm{ref}} > 4 \cdot 10^{-2}$. This was chosen as the criterion for the lattice to be frozen, in which case we moved on to a larger lattice and continued. To deal with the inefficiencies of the random element generation for the Fibonacci lattice $20,000$ thermalization iterations were run, with $1000$ measuring iterations.

The results can be found in figure \ref{plot:fibPhaseScan}. As can be seen by the lack of hot starting configuration data points for $ n > 24 $, the higher amount of thermalization iterations did not help much there. The results towards the larger lattice sizes however seem to have significantly improved, reading now $\beta_{\textrm{ph.}}(F_{64}) > 5.7$ and $\beta_{\textrm{ph.}}(F_{128}) > 8.6$. It should be noted that the chosen criterion for a frozen lattice might also play a role here. We went with $P-P_{\textrm{ref}} > 4 \cdot 10^{-2}$ as it seemed to cover all the previously discussed phase transitions reasonably well.\\

Nevertheless, it becomes obvious that the phase transitions for the Fibonacci lattice can be found at significantly higher values of $\beta$ compared to any of the other lattices. To get an estimate of the phase transition for a given lattice size, the function
\begin{align*}
  \beta_{\textrm{ph.}}^{(\textrm{fib.})}(n) = a \sqrt{n} + b \qquad \qquad  \textrm{with} \qquad a &= 0.8532 \pm 0.0037\\
  \textrm{and} \qquad b &= -1.080 \pm 0.027
\end{align*}
was fitted to the cold start data points. With a $\chi_{\textrm{red.}}^2 = 0.37$ this seems to reproduce the data reasonably well. Whether this holds for lattice sizes outside the tested range might however be questionable.

\subsection{Systematic Deviations}
\FloatBarrier

Lastly, we took a more detailed look at some systematic deviations that might occur for the different approximations of \SUTwo. For this we ran another reference simulation for $\beta = 0.1$ and $\beta = 1.0$ with $10^6$ iterations, and then tested the approximations with $10^5$ iterations. The results can be found in figures \ref{plot:systemBeta01} and \ref{plot:systemBeta10}. For $\beta=0.1$ only $C_5$ and $C_{16}$ as well as $F_{8}$ and $F_{16}$ significantly deviate from the continuous case.

For $\beta=1$ also $\overline{T}$ and $C_8$ started to show significant deviations. The slightly higher deviation for $V_4$ turned out to just be a statistical fluctuation, as verified by running the same test for different values of $\beta$.\\

\begin{figure}[!hbt]
 \centering
 \input{systemCheck0.1.pgf}
 \caption{Systematic deviations for $\beta = 0.1$}
 \label{plot:systemBeta01}
\end{figure}
\begin{figure}[!hbt]
 \centering
 \input{systemCheck1.0.pgf}
 \caption{Systematic deviations for $\beta = 1.0$}
 \label{plot:systemBeta10}
\end{figure}

The deviations for $F_8$ and $F_{16}$ at $\beta=0.1$ are quite noteworthy, as contrary to the behavior discussed for phase transitions the eight vertices of $C_{16}$ are a lot closer to the reference data set, while the $16$ vertices of $C_8$ do not even show any significant deviation.

At $\beta=1$ however, one is already approaching the phase transitions for these lattices. Therefore, here the Fibonacci lattices are closer to the reference data set, for similar vertex count.\\

This can be explained as follows. For very small values of $\beta$, the acceptance rate will be very high. This means the behavior of the Monte Carlo algorithm is very close to just picking random elements from the gauge set, and averaging them. As most of the lattices are constructed with some symmetry in mind, they are \emph{balanced} in the sense that
\begin{align*}
 \sum_{v \in \{\textrm{vertices}\}} v = 0
\end{align*}
holds for them. This however is not the case for the Fibonacci lattice, which is why for small $\beta$, such a lack of \emph{balance} shows up as a systematic deviation. With increasing the size of the Fibonacci lattices, these effects however decrease.

\FloatBarrier
